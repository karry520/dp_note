\section{CNN}
\subsection{卷积}
\subsection{卷积层的前向传播}
为简单起见,考虑单通道的情况
\begin{equation}
	\boldsymbol{X} * \boldsymbol{K} = \boldsymbol{Y}
\end{equation}
即
\begin{equation}
	\begin{pmatrix}
	x_{11}&  x_{12}& x_{13} \\ 
	x_{21}&  x_{22}&  x_{23}\\ 
	x_{31}&  x_{32}& x_{33}
	\end{pmatrix} * 
	\begin{pmatrix}
	k_{11}& k_{12} \\ 
	k_{21}& k_{22}
	\end{pmatrix} =
	\begin{pmatrix}
	y_{11}& y_{12} \\ 
	y_{21}& y_{22}
	\end{pmatrix} 
\end{equation}
这里
\begin{equation}
	\begin{aligned}
	\begin{pmatrix}
	y_{11}\\ 
	y_{12}\\ 
	y_{21}\\ 
	y_{22}
	\end{pmatrix} &= 
	\begin{pmatrix}
	x_{11}k_{11} +x_{12}k_{12} +x_{21}k_{21} +x_{22}k_{22} \\ 
	x_{12}k_{11} +x_{13}k_{12} +x_{22}k_{21} +x_{23}k_{22}\\
	x_{21}k_{11} +x_{22}k_{12} +x_{31}k_{21} +x_{32}k_{22}\\
	x_{22}k_{11} +x_{23}k_{12} +x_{32}k_{21} +x_{33}k_{22}
	\end{pmatrix}\\
	&=\begin{pmatrix}
	x_{11} &x_{12} &x_{21} &x_{22}  \\ 
	x_{12} &x_{13} &x_{22} &x_{23} \\
	x_{21} &x_{22} &x_{31} &x_{32} \\
	x_{22} &x_{23} &x_{32} &x_{33} 
	\end{pmatrix} \cdot 
	\begin{pmatrix}
	k_{11}\\ 
	k_{12}\\ 
	k_{21}\\ 
	k_{22}
	\end{pmatrix}
	\end{aligned}
\end{equation}
所以,卷积运算最终转化为矩阵运算。需要对原始的$\boldsymbol{X},\boldsymbol{K},\boldsymbol{Y}$进行变形操作,相应的记作$\boldsymbol{XC},\boldsymbol{KC},\boldsymbol{YC}$。多通道的情况只需要在维度上将操作扩展即可。
\subsection{卷积层的反向传播}
分析$\delta$误差反向传播过程可以简单的记忆为:如果神经网络$l+1$层某个结点的$\delta$误差要传到$l$层,我们就去找到前向传播时$l+1$层的这个结点和第$l$层的哪些结点有关系,权重是多少,那么反向传播时,$\delta$误差就会乘上相同的权重传播回来。

为了书写方便,记
\begin{equation}
	\begin{pmatrix}
		\delta_{11}\\ 
		\delta_{12}\\ 
		\delta_{21}\\ 
		\delta_{22}
	\end{pmatrix}=\nabla \boldsymbol{YC} = 
	\begin{pmatrix}
		\nabla y_{11}\\ 
		\nabla y_{12}\\ 
		\nabla y_{21}\\ 
		\nabla y_{22}
	\end{pmatrix}
\end{equation}
在反向传播中,$\delta$是从后面一层(一般是激活函数层或池化层)传过来的,是一个已知量,在此基础上求$\nabla \boldsymbol{K}, \nabla \boldsymbol{X}$
\begin{enumerate}
	\item 求$\nabla \boldsymbol{K}$
	\begin{equation}
		\nabla \boldsymbol{KC} = \boldsymbol{XC}^T\cdot \nabla \boldsymbol{YC}
	\end{equation}
	$\nabla \boldsymbol{KC}$ 只要reshape一下就可以得到$\nabla \boldsymbol{K}$
	\item 求$\nabla \boldsymbol{X}$
	
	根据反向传播公式,
	\begin{equation}
		\nabla \boldsymbol{XC} = \nabla \boldsymbol{YC} \cdot \boldsymbol{KC}^T
	\end{equation}
	但是,从$\nabla \boldsymbol{XC}$还原到$\nabla \boldsymbol{X}$不是一件容易的事,所以考虑新的计算方式。
	
	根据前向传播
	\begin{equation}
		\begin{pmatrix}
		y_{11}\\ 
		y_{12}\\ 
		y_{21}\\ 
		y_{22}
		\end{pmatrix} = 
		\begin{pmatrix}
		x_{11}k_{11} +x_{12}k_{12} +x_{21}k_{21} +x_{22}k_{22} \\ 
		x_{12}k_{11} +x_{13}k_{12} +x_{22}k_{21} +x_{23}k_{22}\\
		x_{21}k_{11} +x_{22}k_{12} +x_{31}k_{21} +x_{32}k_{22}\\
		x_{22}k_{11} +x_{23}k_{12} +x_{32}k_{21} +x_{33}k_{22}
		\end{pmatrix}
	\end{equation}
	可以计算每个$x_{ij}$的导数
	\begin{equation}
		\begin{aligned}
		\begin{pmatrix}
		\nabla x_{11} \\ 
		\nabla x_{12}\\ 
		\nabla x_{13}\\ 
		\nabla x_{21}\\ 
		\nabla x_{22}\\ 
		\nabla x_{23}\\ 
		\nabla x_{31}\\ 
		\nabla x_{32}\\ 
		\nabla x_{33}
		\end{pmatrix} &= \begin{pmatrix}
			k_{22}0 + k_{21}0 + K_{12}0+k_{11}\delta_{11} \\
			k_{22}0 + k_{21}0 + K_{12}\delta_{11}+k_{11}\delta_{12} \\
			k_{22}0 + k_{21}0 + K_{12}\delta_{12}+k_{11}0 \\
			k_{22}0 + k_{21}\delta_{11} + K_{12}0+k_{11}\delta_{21} \\
			k_{22}\delta_{11} + k_{21}\delta_{12} + K_{12}\delta_{21}+k_{11}\delta_{22} \\
			k_{22}\delta_{12} + k_{21}0 + K_{12}\delta_{22}+k_{11}0 \\
			k_{22}0 + k_{21}0 + K_{12}\delta_{21}+k_{11}0 \\
			k_{22}\delta_{21} + k_{21}\delta_{22} + K_{12}0+k_{11}0 \\
			k_{22}\delta_{22} + k_{21}0 + K_{12}0+k_{11}0 
		\end{pmatrix}\\
		&= \begin{pmatrix}
		0 & 0 & 0 & \delta_{11} \\ 
		0 & 0 & \delta_{11} & \delta_{12} \\ 
		0 & 0 & \delta_{12} & 0 \\ 
		0 & \delta_{11} & 0 & \delta_{21} \\ 
		\delta_{11} & \delta_{12} & \delta_{21} & \delta_{22} \\ 
		\delta_{12} & 0 & \delta_{22} & 0 \\ 
		0 & \delta_{21} & 0 & 0 \\ 
		\delta_{21} & \delta_{22} & 0 & 0 \\ 
		\delta_{22} & 0 & 0 & 0
		\end{pmatrix} \cdot 
		\begin{pmatrix}
			k_{11}\\ 
			k_{12}\\ 
			k_{21}\\ 
			k_{22}
		\end{pmatrix}
		\end{aligned}
	\end{equation}
	设上面三个矩阵分别为$\nabla \boldsymbol{X}^{'},\nabla \boldsymbol{Y}^{'},\nabla \boldsymbol{K}^{'}$,即$\nabla \boldsymbol{X}^{'}=\nabla \boldsymbol{Y}^{'}\cdot \nabla \boldsymbol{K}^{'}$。从而可见
	\begin{equation}
		\nabla \boldsymbol{X} = 
		\begin{pmatrix}
		\nabla x_{11}&\nabla  x_{12}&\nabla x_{13} \\ 
		\nabla x_{21}&\nabla  x_{22}&\nabla  x_{23}\\ 
		\nabla x_{31}&\nabla  x_{32}&\nabla x_{33}
		\end{pmatrix} = \begin{pmatrix}
		0 & 0 & 0 & 0 \\ 
		0 & \delta_{11} & \delta_{12} & 0 \\ 
		0 & \delta_{21} & \delta_{22} & 0 \\ 
		0 & 0 & 0 & 0
		\end{pmatrix} *
		\begin{pmatrix}
		k_{22} & k_{21} \\ 
		k_{12} & k_{11}
		\end{pmatrix} 
	\end{equation}
	这是一个卷积运算。不同的是对$\nabla \boldsymbol{Y}$进行卷积,从后向前卷积,有的文章称为逆向卷积。
\end{enumerate}

\subsection{池化层及池化层的反向传播}